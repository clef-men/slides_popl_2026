\section{\Zoo overview}

\begin{frame}{\Zoo}
\centering
\Huge
\begin{tabular}{ccc}
      \begin{tabular}{c}
          \includegraphics[scale=.29]{images/ocaml.png}
        \\
          \includegraphics[scale=.12]{images/dune.png}
      \end{tabular}      
    &
      \begingroup
        \renewcommand{\arraystretch}{.8}
        \begin{tabular}{c}
            \Large\c{ocaml2zoo}
          \\
            $\step{\hskip3.5em}$
        \end{tabular}
      \endgroup
    &
      \begingroup
        \renewcommand{\arraystretch}{1.8}
        \begin{tabular}{c}
            \includegraphics[scale=.7]{images/iris.pdf}
          \\
            \textcolor{color1}{\fontsize{40}{0}\selectfont Zoo}
          \\
            \includegraphics[scale=.24]{images/rocq.png}
        \end{tabular}
      \endgroup
\end{tabular}
\end{frame}

\begin{frame}[fragile]{\Zoo in practice}
\colorlet{domainslib}{blue}
\colorlet{scheduler}{cyan}
\colorlet{saturn}{red}
\colorlet{stack}{orange}
\centering
\begin{tabular}{ccc}
    \begin{minipage}{0.4\textwidth}\dirtree{%
      .1 project.
        .2 dune-project.
        .2 lib.
          .3 \textcolor{domainslib}{domainslib}.
            .4 dune.
            .4 \textcolor{scheduler}{scheduler.ml}.
            .4 scheduler.mli.
          .3 \textcolor{saturn}{saturn}.
            .4 dune.
            .4 \textcolor{stack}{stack.ml}.
            .4 stack.mli.
    }\end{minipage}
  &
    $\LARGE\step{\hskip2em}$
  &
    \begin{minipage}{0.4\textwidth}\dirtree{%
      .1 theories.
        .2 \textcolor{domainslib}{domainslib}.
          .3 \textcolor{scheduler}{scheduler\_\_code.v}.
          .3 \textcolor{scheduler}{scheduler\_\_types.v}.
        .2 \textcolor{saturn}{saturn}.
          .3 \textcolor{stack}{stack\_\_code.v}.
          .3 \textcolor{stack}{stack\_\_types.v}.
    }\end{minipage}
\end{tabular}
\vfill
\begin{center}
  \Large\c{\$ ocaml2zoo project theories}
\end{center}
\end{frame}

%\begin{frame}[fragile]{\Zoo in practice}
%\colorlet{domainslib}{blue}
%\colorlet{scheduler}{cyan}
%\centering
%\begin{tabular}{ccc}
%    \begin{minipage}{0.4\textwidth}\dirtree{%
%      .1 project.
%        .2 dune-project.
%        .2 lib.
%          .3 \textcolor{domainslib}{domainslib}.
%            .4 dune.
%            .4 \textcolor{scheduler}{scheduler.ml}.
%            .4 scheduler.mli.
%    }\end{minipage}
%  &
%    $\LARGE\step{\hskip2em}$
%  &
%    \begin{minipage}{0.4\textwidth}\dirtree{%
%      .1 theories.
%        .2 \textcolor{domainslib}{domainslib}.
%          .3 \textcolor{scheduler}{scheduler\_\_code.v}.
%          .3 \textcolor{scheduler}{scheduler\_\_types.v}.
%    }\end{minipage}
%\end{tabular}
%\vfill
%\begin{center}
%  \Large\c{\$ ocaml2zoo project theories}
%\end{center}
%\end{frame}

\begin{frame}[fragile]{\Zoo in practice}
\centering
\vspace{2mm}
\begin{ocamlcode}
let rec push t v =
  let old = Atomic.get t in
  let new_ = v :: old in
  if not (Atomic.compare_and_set t old new_) then (
    Domain.cpu_relax () ;
    push t v
  )
\end{ocamlcode}
\vspace{-10mm}
\hfill \includegraphics[scale=.1]{images/ocaml.png}
\medskip
\hrule
\medskip
\hfill \includegraphics[scale=.17]{images/rocq.png}
\vspace{-7mm}
\begin{coqcode}
Definition stack_push : val :=
  rec: "stack_push" "t" "v" =>
    let: "old" := !"t" in
    let: "new" := ‘Cons( "v", "old" ) in
    if: ~ CAS "t" "old" "new" then (
      domain_cpu_relax () ;;
      "stack_push" "t" "v"
    ).
\end{coqcode}
\end{frame}

\begin{frame}[fragile]{\Zoo in practice}
\centering
\Large
\begin{minipage}[t]{.5\textwidth}
\begin{overprint}
\onslide<1>
\begin{coqcode}
Lemma stack_push_spec t ι v :
  <<< 
    stack_inv t ι
  | ∀∀ vs, stack_model t vs
  >>>
    stack_push t v @ ↑ι
  <<< 
    stack_model t (v :: vs)
  | RET (); True
  >>>.
Proof. ... Qed.
\end{coqcode}
\onslide<2>
\begin{minted}[escapeinside=??]{coq}
Lemma stack_push_spec t ι v :
  <<< 
    stack_inv t ι
  | ?\hl{∀∀ vs, stack\_model t vs}?
  >>>
    stack_push t v @ ↑ι
  <<< 
    ?\hl{stack\_model t (v :: vs)}?
  | RET (); True
  >>>.
Proof. ... Qed.
\end{minted}
\end{overprint}
\end{minipage}
\hfill
\begin{minipage}[t]{.4\textwidth}
\centering
\strut \\
\strut \\
\strut \\
\strut \\
\strut \\
\c{stack\_push} is \\
\emph{linearizable}
\end{minipage}
\end{frame}

% following a pragmatic approach, we support what's needed in practice, in the concurrent programs we studied
\begin{frame}{\Zoo features}
\Large
\begin{minipage}[t]{.5\textwidth}
\begin{itemize}
  \item Algebraic data types
  \item Records
  \item Mutually recursive functions
  \item Physical equality
  \item Structural equality
\end{itemize}
\end{minipage}
\hfill
\begin{minipage}[t]{.45\textwidth}
\begin{itemize}
  \item Atomic references
  \item \textcolor{color2}{Atomic record fields}
  \item \textcolor{color2}{Atomic arrays}
  \item \textcolor{color2}{Generative constructors}
\end{itemize}
\end{minipage}
\vfill
\begin{itemize}
  \item Prophecy variables  
  \item \Diaframe (basic automation)
\end{itemize}
\end{frame}
